\section{Introduction}

Smartphones are ubiquitous: we are witnessing an astonishing growth in mobile phone subscriptions. The International Telecommunication Union estimates there are nearly 7 billion mobile subscriptions worldwide \cite{Mobiforge14}. Meanwhile, the mobile phones' sensors (e.g., cameras) are advancing and the network bandwidth is constantly increasing. Consequently, every person with a mobile phone can now act as a multi-modal sensor, collecting and sharing various types of high-fidelity spatiotemporal data instantaneously (e.g., picture, video, audio, location, time, speed, direction, and acceleration).

Exploiting this large crowd of potential workers and their mobility, a new mechanism for efficient and scalable data collection has emerged: Spatial Crowdsourcing \cite{Kazemi12}. Spatial crowdsourcing requires workers (e.g., willing individuals) to perform a set of tasks by physically traveling to certain locations at particular times. Spatial crowdsourcing has applications in numerous domains such as citizen-journalism, tourism, intelligence, disaster response and urban planning. With spatial crowdsourcing, the requester issues a spatiotemporal query to a spatial crowdsourcing server (SC-Server). Consequently, the SC-server distributes the query among the available workers in the vicinity of the events (in time and space). Once the workers document their corresponding events with their mobile phones, the results are sent back to the requester.

Finding the best worker to assign a task in real-time becomes a major challenge once there is a large number of tasks and workers. On one hand, an SC-Server must consider the schedule of every worker and assign a task to a worker who can reach to and perform the task before its expiration time. On the other hand, the process must be performed frequently and in real-time as new tasks and workers become available or as tasks are completed (or expired) and workers leave the system.

Several existing approaches \cite{Kazemi12,Deng13,Cheng15,Li15,Alfarrarjeh15} have focused on the task assignment problem in SC.  To achieve scalability, instead of a real-time assignment upon the task's arrival, Kazemi and Shahabi \cite{Kazemi12} wait for multiple tasks to arrive and process them together. As a result, the SC-Server will have enough time to process the current batch of tasks while it is waiting for the next batch to arrive. Alfarrarjeh et. al. \cite{Alfarrarjeh15} distribute the arriving tasks on multiple SC-Servers but still fall short of a real-time task assignment. In addition to not being real-time, neither studies consider the schedule of the workers upon assignment. On the contrary, Deng et. al. \cite{Deng13} consider the problem of scheduling assigned tasks for a single worker. However, scheduling is done once all the tasks have already been assigned to the worker. Two recent studies \cite{Li15,Deng15} assign and schedule tasks simultaneously. However \cite{Li15} only performs it for a single worker and \cite{Deng15} does not perform it in real-time. To the best of our knowledge, real-time assignment and scheduling is a problem yet to be solved.

Towards this end, we propose Auction-SC; an auction-based framework for real-time task assignment and scheduling. In this framework, we decentralize the scheduling problem by utilizing the workers. The SC-Server broadcasts a task to the workers upon the task's arrival. Each worker \footnote{Hereafter, we use the term "worker" interchangeably to refer to both the human worker and the software running on his/her mobile device unless clear distinction is needed.} submits a bid for that task based on its current schedule and location. To compute its bid, each worker has to consider only its own schedule so the bid computation phase can be done in real-time. Once every worker submits its bid to the SC-Server, the server will select the highest bid as the winner and assign the task to the corresponding worker.

We introduce a branch-and-bound scheduling algorithm where for each new task, the worker performs an exhaustive search to find out whether it can fit the incoming task into its schedule or not. We show that at each point of time the number of remaining tasks for each worker (the number of tasks that the worker has scheduled and not completed yet) is in a range that even the branch-and-bound algorithm can be completed in real-time. However in our experiments, we show that even replacing the branch-and-bound algorithm with a polynomial time approximate algorithm, will not affect the quality of the assignment significantly.

In addition to the branch-and-bound algorithm, we propose several simple but effective spatiotemporal bidding techniques. Moreover, we propose a more complex bidding technique that takes into consideration the spatial distribution of the tasks seen so far. The key idea is that having more workers in areas with more tasks can increase the quality of the assignment. This consideration increases the complexity of the bid computation phase such that it may impact the scalability of the framework. However, we show that the SC-Server can still manage a throughput of \textasciitilde 200 tasks per second with this complex bidding technique.

As mentioned earlier, the SC-Server does not know about the time and location of future tasks and workers, until they arrive. Therefore, it is impossible for the SC-Server to perform a globally optimal assignment. In order to better evaluate the quality of the real-time assignments made by the Auction-SC, for the first time we propose an algorithm for finding the globally optimal assignment in a spatial crowdsourcing environment. We assume there exists a clairvoyant which knows exactly at what time and which location new tasks and workers will arrive and for how long they will be available. We prove that the clairvoyant algorithm is NP-Complete and cannot be performed on large workloads with thousands of tasks and workers. However, in our experiments, for small workloads, we compare the results of Auction-SC with the globally optimal assignment.

We conduct many experiments on both real world and synthetically generated workloads to evaluate different aspects of Auction-SC. For our real world data we use a data set of hundreds of thousands of geo-tagged Flickr images. We also generated synthetic workloads with different spatial and temporal properties and study the effect of various spatiotemporal parameters in a spatial crowdsourcing environment. At the end we analyze the scalability of our framework. We show that with a centralized approach it is not possible to perform real-time assignments that considers the spatiotemporal characteristics of SC. However, by utilizing the auction based framework we enable real-time task assignment and scheduling at scale. 

The remainder of this paper is organized as follows. In \cref{sec:prelim} we formally define the task assignment problem in SC and prove it is NP-Complete. We review the related work in \cref{sec:related}. Following we propose our offline clairvoyant algorithm in \cref{sec:exactalgo} and analyze its time complexity. We introduce Auction-SC in \cref{sec:onlinealgo} and propose several bidding rules within the auction-based framework. We show the results of our experiments on both real world and synthetic data in \cref{sec:experiments} and end the paper with some discussions (\cref{sec:discuss}) and guidelines for future work (\cref{sec:future}).