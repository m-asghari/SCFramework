\section{Discussion}
\label{sec:discuss}

In this section we discuss some of the potential drawbacks of Auction-SC and how they can be addressed.

As mentioned when introducing the framework, the server does not need to keep track of the exact location and the schedule of each worker. Thus, the workers may submit incorrect bids on purpose to increase their chance of winning the auction. Later the worker can decide which assigned task it actually would like to complete. The problem becomes more severe once we reward the workers upon task completion. To overcome this problem we can utilize our server with a trust model \cite{Ye15}. We can train and build the trust model over time and use it to better evaluate the bids from each worker.

In our framework the Auction-SC server will broadcast every incoming task to all workers. The bid computation is a background process running on the worker's smartphone. Therefore, broadcasting every task to every worker does not mean we will be overwhelming the workers with incoming tasks. Nevertheless, if the frequent bid computation starts causing problems such as battery drains, the server can only send each incoming task to a subset of workers.
For example, the server can build a space partitioning structure, e.g., a Grid index, and only broadcast tasks to workers in corresponding partitions.  This approach strikes a compromise between the server's load (e.g., maintaining the structure) and the number of broadcasts. Note that if location privacy is a concern, this paradigm can be implemented in a privacy-preserving manner following  the technique  proposed in \cite{To14}.