\section{Related Work}

%The first class of similar problems is called Volunteered Geographic Information (VGI) whose goal is to generate geographical information provided voluntarily by individuals. Most real-world examples of VGI are limited to adding contents to a pre-built map (Wikimapia, 2006)(Google Map Maker, 2008)(Open Street Map, 2004). Contrary to spatial crowdsourcing, with VGI applications, there’s no need for the worker to be present at any specific location in order to perform the task. As the name suggests, VGI is more self-incentive whereas in spatial crowdsourcing workers are assigned tasks by the SC-server.

%Another class of similar problems, called participatory sensing, exploits mobile users to collect and share data using their sensor-equipped phones for a given campaign. Some applications of participatory sensing in the real-world include traffic monitoring (Univ of California Berkeley, 2008)(CENS, 2008)(Hull, Bychkovsky, Zhang, Chen, Goraczko, Miu, Shih, Balakrishnan and Madden, 2006)(Mohan, Padmanabhan and Ramjee, 2008), urban planning (CENS, 2008)(Brabham, 2009), disaster response (Goodchild and Gennon, 2010), and urban air pollution (Paulos, Honicky and Goodman, 2007)(CENS, 2008). Most studies on participatory sensing focus on small campaigns with limited number of workers. However, with spatial crowdsourcing the focus is on devising a scalable, generic and multi-purpose crowdsourcing framework, similar to Amazon Mechanical Turk, but spatial, where multiple campaigns can be handled simultaneously. Therefore, the main challenge with spatial crowdsourcing is to devise an efficient approach to assign tasks to workers given the large scale and dynamism of the environment.

Some recent studies in spatial matching \cite{Wong07,Long13} do focus on efficiency and use the spatial features of the objects for more efficient assignment. These studies assume a global knowledge about the locations of all objects exists a priori and the challenge comes from the complexity of spatial matching. However, spatial crowdsourcing differs due to the dynamism of tasks and workers (i.e., tasks and workers come and go without our knowledge), thus the challenge is to perform the task assignment at a given instance of time with the goal of global optimization across all times. Moreover, the fact that workers need to travel to task locations causes the landscape of the problem to change constantly. This will add another layer of dynamism to spatial crowdsourcing that makes it a unique problem.

One can consider the task assignment problem in spatial crowdsourcing as a matching problem between tasks and workers, which makes it similar to the classic matching problem \cite{Gibbons85,Avis83}. In particular, the online matching problem \cite{Kalyanasundaram93, Kalyanasundaram00} is the most relevant variation to spatial crowdsourcing as it captures the dynamism of tasks arriving at different times. However, once the number of tasks assigned to one worker is more than one, the online matching problem cannot capture the true cost of performing tasks. More specifically, once you assign a set of tasks to a worker, the cost of executing this set is the distance of the shortest path that starts from the worker’s current location and goes through the locations of all the assigned tasks. On the other hand, with online b-matching \cite{Kalyanasundaram00} the overall cost for one worker would be the sum of the distances between the worker and each assignedtask.

Modeling the assignment cost as the shortest path a worker has to take to visit the locations of multiple tasks, brings another class of problems to attention. In this context the assignment problem in spatial crowdsourcing becomes similar to the Traveling Salesman Problem (TSP) \cite{Lawler85} and the Vehicle Routing Problem (VRP) \cite{Toth02}. The online versions of both TSP and VRP have been studied to some extent where new locations to visit are revealed incrementally. Since there is only one salesman in the standard version of TSP, here we focus on VRP. Different variations of VRP have been studied, yet there are still some differences between task assignment in spatial crowdsourcing and these variations. In VRP, a server can pay a penalty and deny visiting a location; however, in spatial crowdsourcing the goal is to maximize the number of assigned tasks so the worker does not have the option of denying a task. Furthermore, in VRP, all servers start from the same depot where in spatial crowdsourcing each worker can have a different starting location. Moreover, in VRP we have a fixed number of servers whereas in spatial crowdsourcing the same type of dynamism for tasks can apply to the workers. That is, workers can be added (removed) to (from) the system at any time.
