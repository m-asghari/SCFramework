\section{Introduction}

Smartphones are ubiquitous: we are witnessing an astonishing growth in mobile phone subscription that has surpassed 4.35 billion worldwide at the end of 2009 (ABI Research, 2010), reached 6 billion in 2011, which is 87\% of the world population (Mobithink, 2013), and by 2013, Gartner predicted that mobile phones would overtake PCs as the most common Web access device worldwide. Meanwhile, the mobile phone’s bandwidth is constantly increasing: from 2.5G (up to 384Kbps) to 3G (up to 14.7Mbps) and recently 4G (up to 100 Mbps) (Sauter, 2011). Hence, the multiplication of the above two factors plus the constant progress and increase of smartphone’s sensors (e.g., video cameras) suggest an exponential growth in data collection and sharing by smart phones. That is, every person with a mobile phone can now act as a multi-modal sensor collecting and sharing various types of high-fidelity spatiotemporal data instantaneously (e.g., picture, video, audio, location, time, speed, direction, and acceleration).

Exploiting this large volume of potential users and their movability, a new mechanism for efficient and scalable data collection has emerged: “spatial crowdsourcing”(Kazemi and Shahabi, 2012). Spatial crowdsourcing requires workers (e.g., willing individuals) to perform a set of tasks by physically traveling to certain locations at particular times. Spatial crowdsourcing has applications in numerous domains such as journalism, tourism, intelligence, disaster response and urban planning. To illustrate, consider a disaster-response scenario depicted in Figure 1, where Red Cross (i.e., requester) is interested in collecting pictures and videos of disaster areas from various locations of a city. With spatial crowdsourcing, the requester issues a query to a spatial crowdsourcing server (SC-server). Consequently, the SC-server distributes the query among the available workers in the vicinity of the events. Once the workers document their events with their mobile phones, the results are sent back to the requester.

The main challenges of spatial crowdsourcing are due to the large-scale, ad-hoc and dynamic nature of the workers and tasks.  First, to continuously match thousands of spatial crowdsourcing campaigns, where each campaign consists of many spatiotemporal tasks, with millions of workers, an SC-Server must be able to run efficient task-assignment strategies that can scale.  Second, the task-assignment must be performed frequently and in real-time as new tasks and workers become available or as tasks are completed (or expired) and workers leave the system.  Third, while in small campaigns, the workers may be known and trusted, with spatial crowdsourcing, the workers cannot always be trusted. In fact, some skeptics of crowdsourcing go as far as calling it a garbage-in-garbage-out system due to the issue of trust. Finally, individuals with mobile devices need to physically travel to the specified locations of interest. An adversary with access to individual whereabouts can infer sensitive details about a person (e.g., health status and political views); thus, protecting worker location privacy is an important concern in spatial crowdsourcing.
