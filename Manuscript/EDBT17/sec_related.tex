\section{Related Work}
\label{sec:related}

Spatial crowdsorcing research focuses on the task assignment \cite{Kazemi12,Alfarrarjeh15,Cheng15,Deng13,Li15,Chen15,Deng15}, trust \cite{Kazemi13,Cheng15} and privacy issues \cite{To14}. Kazemi and Shahabi \cite{Kazemi12} formulate task assignment in spatial crowdsourcing as a matching problem with the primary objective of maximizing the number of matched tasks, and Alfarrarjeh et al. \cite{Alfarrarjeh15} scaled up the matching algorithm in a distributed setting. Reliable task assignment addressing trust issues in spatial crowdsourcing have been studied in \cite{Kazemi13,Cheng15}. In \cite{Cheng15} Cheng et.al. proposed a partitioning heuristic which recursively divides tasks and workers via KMeans clustering to improve the assignment efficiency. All these studies utilize the batch scheme. The methods used in \cite{Kazemi12,Cheng15,Alfarrarjeh15} cannot be applied to a single task for real-time assignment. Furthermore, neither of these studies consider the schedule of a worker when matching tasks and workers. In their settings a worker is able to complete a task if it is matched to it which is not the same in our case. On the other hand, Deng et. al. \cite{Deng13} studied the scheduling problem from the worker's perspective and developed both exact and heuristic algorithms to help the worker find the best schedule for all the tasks assigned to it. However, their work assumes that each worker has been pre-assigned with some tasks, and thus only deals with scheduling tasks for a single worker. Recent studies have focused on simultaneous task assignment and scheduling \cite{Li15,Deng15,Cheng15}. In \cite{Li15}, Li et. al. perform a real-time task assignment for a \emph{single} worker. Knowing the worker's destination, they recommend a route to the worker to be able to complete as many tasks as possible on its way to the destination point. Deng et. al. \cite{Deng15} and Chen at. al \cite{Chen15} perform assignment and scheduling for multiple workers. However, similar to \cite{Kazemi12,Cheng15,Alfarrarjeh15} they tackle the assignment problem by batching tasks' and workers' arrivals and departures, and then perform matching and scheduling periodically (e.g., every 10 minutes) for each batch.

Our work is also related to some combinatorial optimization problems such as Vehicle Routing Problem(VRP) \cite{Braysy05}. The general setting of VRP is to serve a number of customers with a fleet of vehicles and the objective is to minimize the total travel cost of those vehicles. Compared with VRP, with spatial crowdsourcing our objective is to maximize the number of completed tasks, whereas VRP aims to minimize the total travel time. In addition, the spatial workers in our problem setting are not located at one or several fixed depots, and each worker can show up at any unique location. In our setting the spatial tasks are also not guaranteed to be completed by the workers. Finally, in a spatial crowdsourcing platform, we need to provide efficient solutions for potentially millions of tasks and workers. This is different from the solutions in VRP which take hundreds of seconds even for one hundred delivery points.

Auction frameworks have been used in various domains such as in dynamic multi-agent environments \cite{Mehta05,Lagoudakis04} and online advertising \cite{Ghosh10}. Bidding strategies from other domains cannot be used in spatial crowdsourcing as they do not consider the spatial and temporal characteristics of SC. In this paper, we develop bidding strategies that are unique to a spatial crowdsourcing environment. One aspect of auction frameworks that has been widely studied is how to prevent bidders from submitting malicious bids to give themselves unfair advantage \cite{Lavi05}. This problem does not apply to our Auction-SC framework because here, the workers do not directly interact with the SC-Server. Instead, the software on a worker's phone receives tasks, computes bids and submits them to the server. If a task is assigned to a worker, the worker will be notified with an updated schedule through the software. The worker has no control over the software and hence, cannot cheat the system.

%Some recent studies in spatial matching \cite{Wong07,Long13} do focus on efficiency and use the spatial features of the objects for more efficient assignment. These studies assume a global knowledge about the locations of all objects exists a priori and the challenge comes from the complexity of spatial matching. However, spatial crowdsourcing differs due to the dynamism of tasks and workers (i.e., tasks and workers come and go without our knowledge), thus the challenge is to perform the task assignment at a given instance of time with the goal of global optimization across all times. Moreover, the fact that workers need to travel to task locations causes the landscape of the problem to change constantly. This will add another layer of dynamism to spatial crowdsourcing that makes it a unique problem.

%One can consider the task assignment problem in spatial crowdsourcing as a matching problem between tasks and workers, which makes it similar to the classic matching problem \cite{Gibbons85,Avis83}. In particular, the online matching problem \cite{Kalyanasundaram93, Kalyanasundaram00} is the most relevant variation to spatial crowdsourcing as it captures the dynamism of tasks arriving at different times. However, once the number of tasks assigned to one worker is more than one, the online matching problem cannot capture the true cost of performing tasks. More specifically, once you assign a set of tasks to a worker, the cost of executing this set is the distance of the shortest path that starts from the worker’s current location and goes through the locations of all the assigned tasks. On the other hand, with online b-matching \cite{Kalyanasundaram00} the overall cost for one worker would be the sum of the distances between the worker and each assignedtask.

%Modeling the assignment cost as the shortest path a worker has to take to visit the locations of multiple tasks, brings another class of problems to attention. In this context the assignment problem in spatial crowdsourcing becomes similar to the Traveling Salesman Problem (TSP) \cite{Lawler85} and the Vehicle Routing Problem (VRP) \cite{Toth02}. The online versions of both TSP and VRP have been studied to some extent where new locations to visit are revealed incrementally. Since there is only one salesman in the standard version of TSP, here we focus on VRP. Different variations of VRP have been studied, yet there are still some differences between task assignment in spatial crowdsourcing and these variations. In VRP, a server can pay a penalty and deny visiting a location; however, in spatial crowdsourcing the goal is to maximize the number of assigned tasks so the worker does not have the option of denying a task. Furthermore, in VRP, all servers start from the same depot where in spatial crowdsourcing each worker can have a different starting location. Moreover, in VRP we have a fixed number of servers whereas in spatial crowdsourcing the same type of dynamism for tasks can apply to the workers. That is, workers can be added (removed) to (from) the system at any time.
