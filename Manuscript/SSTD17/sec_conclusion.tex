\section{Conclusion and Future Work}
\label{sec:future}

In this paper, we studied the problem of real-time task assignment in spatial crowdsourcing. We showed that neither of the two approaches for task assignment in spatial crowdsourcing, batched assignment and monolithic-SC, can scale as either task matching or task scheduling will become the bottleneck. Therefore, we introduced an auction-based framework in which we split the matching and scheduling responsibilities between the SC-Server and workers, respectively. We showed that by exploiting the spatiotemporal aspects of SC, with our proposed algorithms, the workers will be able to complete up to 30\% more tasks as compared to batched assignment and non-SC approaches.\\

In this paper, we assumed each task can be performed instantaneously, e.g., taking a picture. Once that assumption is relaxed, we will face new challenges with scheduling the tasks. We also assumed each task requires the worker to travel to a single location. However, in other applications the worker may need to visit multiple locations for a single task, e.g., in the Uber application the worker has to pick up a passenger at one location and drop him off at a second location. We plan to extend our Auction-SC framework to incorporate these two features.