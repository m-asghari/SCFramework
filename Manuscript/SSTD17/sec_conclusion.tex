\vspace{-0.1in}
\section{Conclusion and Future Work}
\label{sec:future}
\vspace{-0.1in}

We studied the problem of online task assignment in spatial crowdsourcing. We introduced an auction-based framework in which we split the matching and scheduling responsibilities between the SC-Server and workers, respectively. We compared our framework with three alternative approaches; the scheduling-oblivious-matching (SOM) approach where matching and scheduling are performed separately, the batched approach where the server periodically performs matching and scheduling on a batch of tasks and the online monolithic approach where tasks get processed one at a time but the server performs scheduling for all workers. roueic es h ie nuWe showed that by exploiting the spatiotemporal aspects of SC, with our proposed algorithms, the workers will be able to complete up to 30\% more tasks as compared to the SOM and batched approaches while scaling orders of magnitude higher than both batched and online monolithic-SC methods.

In this paper, we assumed each task can be performed instantaneously, e.g., taking a picture. Once that assumption is relaxed, we will face new challenges with scheduling the tasks. We also assumed each task requires the worker to travel to a single location. However, in other applications the worker may need to visit multiple locations for a single task, e.g., in the Uber application the worker has to pick up a passenger at one location and drop him off at a second location. We plan to extend our Auction-SC framework to incorporate these two features.